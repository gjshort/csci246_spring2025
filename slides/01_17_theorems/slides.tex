\documentclass[10pt]{beamer}

%%%
% PREAMBLE FOR THIS DOC 
%%%
%https://tex.stackexchange.com/questions/68821/is-it-possible-to-create-a-latex-preamble-header
\usepackage{../preambles/beamer_preamble_for_CSCI246}



%%% TRY TO RESHOW TOC AT EACH SECTION START (with current section highlighted)
% Reference: https://tex.stackexchange.com/questions/280436/how-to-highlight-a-specific-section-in-beamer-toc
\newcommand\tocforsect[2]{%
  \begingroup
  \edef\safesection{\thesection}
  \setcounter{section}{#1}
  \tableofcontents[#2,currentsection]
  \setcounter{section}{\safesection}
  \endgroup
}


%%%% HERES HOW TO DO IT CORRECTLY
% FIRST IN .STY FILE, DO
%\usetheme[sectionpage=none]{metropolis}
% THEN AT EACH SECTION DO
%\begin{frame}{Outline}
%  \tableofcontents[currentsection]	
%\end{frame}



%\setbeamertemplate{navigation symbols}{}
%\setbeamertemplate{footline}[frame number]{}


%%%
% DOCUMENT
%%%

\begin{document}

%\maketitle

%% Title page frame
%\begin{frame}
%    \titlepage 
%\end{frame}





\title{Friday 01/17/2025: Theorems}
\author{CSCI 246: Discrete Structures}
\date{}

\begin{frame}
    \titlepage 
\end{frame}



\begin{frame}{Quiz}

Replace each $\red{?}$ with a checkmark \greencheck if the combination of truth values for propositions A and B is \textit{possible} under the given logical connective.  Replace it with a \redx if the combination is \textit{impossible}.

\begin{table}
\centering
\begin{tabular}{cc|ccc}
\multicolumn{2}{c}{\textbf{Propositions}} & \multicolumn{3}{c}{\textbf{Logical Connectives}} \\
A  & B & if A then B & if B then A & A if and only if B \\
\hline 
T & T & \red{?} & \red{?} & \red{?}\\
T & F &\red{?}  & \red{?} & \red{?} \\
F & T & \red{?} & \red{?} & \red{?} \\
F & F & \red{?} &\red{?}  & \red{?}
\end{tabular}
\end{table}
\vfill \vfill \vfill 
\pause 
\alertstar \alert{See whiteboard for solution.}
\end{frame}


\begin{frame}{Propositional logic}
\label{slide:prop_logic}
\small 
\begin{table}
\centering
\begin{tabular}{cc|ccc}
\multicolumn{2}{c}{\textbf{Original propositions}} & \multicolumn{3}{c}{\textbf{New propositions}} \\
A  & B & if A then B & if B then A & A if and only if B \\
\hline 
T & T & \green{T}  & \green{T} & \green{T}\\
T & F & \red{F} & \green{T} &  \red{F}  \\
F & T & \green{T}  &  \red{F}  &  \red{F}  \\
F & F & \green{T} & \green{T} & \green{T}
\end{tabular}
\end{table}

\begin{itemize}
\item The column headings show 3 new propositions, formed from the original propositions by \textbf{logical connectives}.
\item The first two columns combined with one remaining column gives the \textbf{truth table} for that logical connective. 
\item Each logical connective can be thought of as a \textbf{function} or mapping from $\set{T,F} \times \set{T,F} \to \set{T,F}$. 
\item There are other such functions (\texttt{and, or, xor}, etc.), some of which were discussed in the text.
\item The study of how to combine and change propositions under logical connectives to form more complex propositions is called \textbf{propositional logic}. 
\end{itemize}


\end{frame}


\begin{frame}[standout]

\alert{Group work!}
\vfill
Students are randomly assigned into groups of 3 on the next slide.
\vfill 
Each group gets $\half$ of a white board.
\vfill
If the  $\half$ white board is inconvenient, feel free to write on a window! 
\end{frame}

\begin{frame}
\footnotesize
Group 1: timothy.true,conner.reed1,connor.mizner\\
Group 2: jacob.ruiz1,evan.barth,evan.schoening\\
Group 3: matthew.nagel,connor.graville,adam.wyszynski\\
Group 4: lynsey.read,connor.yetter,ryan.barrett2\\
Group 5: caitlin.hermanson,james.brubaker,peter.buckley1\\
Group 6: derek.price4,alexander.goetz,jacob.ketola\\
Group 7: tristan.nogacki,jeremiah.mackey,michael.oswald\\
Group 8: nicholas.harrington1,aaron.loomis,joseph.windmann\\
Group 9: samuel.rollins,zeke.baumann,samuel.hemmen\\
Group 10: erik.moore3,colter.huber,devon.maurer\\
Group 11: jonas.zeiler,luke.donaldson1,carver.wambold\\
Group 12: jett.girard,carsten.brooks,justice.mosso\\
Group 13: luka.derry,nolan.scott1,owen.obrien\\
Group 14: anthony.mann,samuel.mosier,blake.leone\\
Group 15: yebin.wallace,peyton.trigg,emmeri.grooms\\
Group 16: julia.larsen,tyler.broesel,sarah.periolat\\
Group 17: bridger.voss,jack.fry,micaylyn.parker\\
Group 18: jacob.shepherd1,ethan.johnson18,joseph.triem\\
Group 19: cameron.wittrock,lucas.jones6,jada.zorn\\
Group 20: reid.pickert,delaney.rubb,alexander.knutson\\
Group 21: griffin.short,jakob.kominsky,john.fotheringham\\
Group 22: mason.barnocky,william.elder1,kaden.price\\
Group 23: pendleton.johnston,joseph.mergenthaler
\end{frame}


\begin{frame}{Group exercises}
\small 
\begin{enumerate}
 % Scheinerman
	\item It is a common mistake to confuse the following two statements (i) If A, then B and (ii) If B, then A. Find two conditions A and B such that statement (i) is true but statement (ii) is false. Then find two conditions A and B such that both statements are true. 
% Erciyes 
	\item  Two propositions are considered \textit{equivalent} if they have the same truth table values. Show that the biconditional $A \iff B$ is equivalent to $(A \implies B) \, \texttt{and} \, (B \implies A)$.
 % Scheinerman
	\item Consider these two statements: (i) If A, then B, (ii) If (not B), then (not A).  Under what circumstances are these statements true?  When are they false? Explain whether these statements are identical or not. \alert{[Note: (ii) is called the \textbf{contrapositive} of (i).]}
 % Norman Swartz
	\item (Challenge problem, from philosopher Norman Swartz.)  Is the following statement true or false, and why? \textit{A's-being-a-necessary-condition-for-B is both a necessary and sufficient condition for B's-being-a-sufficient-condition-for-A.}
\end{enumerate}
	
\end{frame}


\begin{frame}{Question 1: Solution}
$A \implies B$ but $B \centernot\implies A$: \\
A = I lived in Los Angeles \\
B = I lived in California. \\
\vfill 
$A \iff B$: \\
A = Valentine's Day is this month \\
B = This month is February. 
\end{frame}

\begin{frame}{Question 2: Solution}
\footnotesize
 The truth table for the ``\texttt{and}" operator (also written $\land$) is given by  
\begin{center}
\begin{tabular}{cc|c}
\multicolumn{2}{c}{\textbf{Original propositions}} & \multicolumn{1}{c}{\textbf{New propositions}} \\
X & Y & $X \land Y$ \\
\hline 
T & T & T \\
T & F & F \\
F & T & F  \\
F & F & F  \\
\end{tabular}
\end{center}

Now we apply the $\land$ operator to the results of the $\implies$ and $\impliedby$ operators.
 
\begin{table}
\centering
\begin{tabular}{cc|ccc}
\multicolumn{2}{c}{\textbf{Orig. props.}} & \multicolumn{3}{c}{\textbf{New props.}} \\
A & B & $\overbrace{A \implies B}^{X}$  & $\overbrace{B \implies A}^{Y}$& $\overbrace{(A \implies B) \land  (B \implies A)}^{X \land Y}$ \\
\hline 
T & T & \green{T}  & \green{T} & \green{T}\\
T & F & \red{F} & \green{T} &  \red{F}  \\
F & T & \green{T}  &  \red{F}  &  \red{F}  \\
F & F & \green{T} & \green{T} & \green{T}
\end{tabular}
\end{table}
%
Note that $(A \implies B) \land  (B \implies A)$ gives the same results as $A \iff B$ as on Slide \ref{slide:prop_logic}.

\end{frame}


\begin{frame}{Question 3: Solution}
\footnotesize
Recall from Slide \ref{slide:prop_logic} that the truth table for the $\implies$ operator is given by  
\begin{center}
\begin{tabular}{cc|c}
X & Y & $\overbrace{X \implies Y}^{\text{If X, then Y}}$ \\
\hline 
T & T & T \\
T & F & F \\
F & T & T  \\
F & F & T  \\
\end{tabular}
\end{center}

Now we apply the $\implies$ operator to the results of the "\texttt{not}" operator (also written $\lnot$).
 
\begin{table}
\centering
\begin{tabular}{cc|ccc}
\multicolumn{2}{c}{\textbf{Orig. props.}} & \multicolumn{3}{c}{\textbf{New props.}} \\
A & B & $\overbrace{\lnot A}^{Y}$  & $\overbrace{\lnot B}^{X}$& $\overbrace{\lnot B \implies \lnot A)}^{X \implies Y}$ \\
\hline 
T & T & \green{T}  & \green{T} & \green{T}\\
T & F & \red{F} & \green{T} &  \red{F}  \\
F & T & \green{T}  &  \red{F}  &  \green{T}  \\
F & F & \green{T} & \green{T} & \green{T}
\end{tabular}
\end{table}
%
Note that $\lnot B \implies \lnot A$ gives the same results as $A \implies B$ as on Slide \ref{slide:prop_logic}.
\vfill 
\pause 
\alertstar \alert{Remark: We've shown that a proposition is logically equivalent to its contrapositive.  So what?  Sometimes it's easier to verify the contrapositive version.}
\end{frame}

\begin{frame}{Question 4: Solution}
The simplest way to see this is as follows:
\begin{itemize}
\item A's-being-a-necessary-condition-for-B can be expressed as $B \implies A$.
\item B's-being-a-sufficient-condition-for-A can be expressed as $B \implies A$. 
\item In other words, both propositions are the same: $B \implies A$. And a proposition is always necessary and sufficient for itself.  
\end{itemize}

\end{frame}


\end{document}
