

%%%%%%%%%%%%%%%%%%%%%%%%%%%%%%%%%%%%%%%%%
% Inzane Syllabus Template
% LaTeX Template
% Version 1.2 (8.22.2019)
%
% This template has been downloaded from:
% http://www.LaTeXTemplates.com
%
% Original author:
% Carmine Spagnuolo (cspagnuolo@unisa.it) with major modifications by 
% Zane Wolf (zwolf.mlxvi@gmail.com)
%
% I (Zane) have left a lot of instructions both in the .tex file and the .cls file that can guide you to customize this document to suite your tastes and requirements. Here is a brief guide: 
%  - Changing the Main Color: .cls line 39
%  - Adding more FAQs: .cls line 126 and .tex line 99
%  - Adding TA emails: uncomment .cls lines 220 & 224 and .tex lines 85 and 89
%  - Deleting the FAQ sidebar entirely: .tex line 188
%  - Removing the Lab/TA Info and placing a brief Overview/About section in their place:        uncomment .tex line 91 and .cls line 227, and comment .cls lines for the LAB/TA info        that you no longer want (c. lines 184-227)

%
% I am also happy to help with crafting/designing modifications to this template to help suite your personal needs in a syllabus. Feel free to reach out! 
%
% License:
% The MIT License (see included LICENSE file)
%
%%%%%%%%%%%%%%%%%%%%%%%%%%%%%%%%%%%%%%%%%

%----------------------------------------------------------------------------------------
%	PACKAGES AND OTHER DOCUMENT CONFIGURATIONS
%----------------------------------------------------------------------------------------

\documentclass[letterpaper]{inzane_syllabus} % a4paper for A4

\usepackage{preamble_syllabus}


\usepackage{booktabs, colortbl, xcolor}
\usepackage{tabularx}
\usepackage{enumitem}
\usepackage{ltablex} 
\usepackage{multirow}

\setlist{nolistsep}

\usepackage{lscape}
\newcolumntype{r}{>{\hsize=0.9\hsize}X}
\newcolumntype{w}{>{\hsize=0.6\hsize}X}
\newcolumntype{m}{>{\hsize=.9\hsize}X}

\renewcommand{\familydefault}{\sfdefault}
\renewcommand{\arraystretch}{2.0}
%----------------------------------------------------------------------------------------
%	 PERSONAL INFORMATION
%----------------------------------------------------------------------------------------

\profilepic{images/rock_stack.png} % Profile picture, if the height of the picture is less than that of the cirle, it will have a flat bottom. 

% To remove any of the following, you need to comment/delete the lines in the .cls file (c. line 186). Commenting/deleting the lines below will produce an error. 

%To add different lines, you will need to create the new command, e.g. \profPhone, in the .cls file c. line 76, and command to create the line in the side bar in the .cls file c. line 186

\classname{Discrete \\ Structures} 
\classnum{CSCI 246} 

%%%%%%%%%%%%%%% PROF INFO
\profname{Dr. Michael  Wojnowicz (Mike)}
\officehours{Office Hrs: Mon \& Wed 1-1:45p, Wed 3:15-4:45p} 
\office{Barnard 352}
\site{https://mikewojnowicz.github.io/} 
\email{michael.wojnowicz@montana.edu}

%%%%%%%%%%%%%%% COURSE  INFO
\classdays{}
\classhours{Class meetings: Mon, Wed, Fri 2:10-3p}
\classloc{Reid 401}
\prereq{Prereq: Calculus, although this is not used.}



%%%%%%%%%%%%%%% TA INFO
\taAname{Fatima Ododo}
\taAofficehours{Mon, Thurs 10-11am}
\taAoffice{Student Success Center (Barnard 259)}
\taAemail{\small fatima.ododo@student.montana.edu}
\taBname{}
\taBofficehours{}
\taBoffice{}
\taBemail{}


%%%%%%%%%%%%%%% TUTOR
\tutorname{Kelly Joyce}
\tutoremail{kelly.joyce1@student.montana.edu}
\tutorofficehours{Thurs 4-6pm}
\tutoroffice{SmartyCats Center (2nd floor Renne)}

%%%% FAQs

%to add more questions or remove this section, go to the .cls file and start with lines comment
%lines 226-250. Also comment out this section as well as line 152(ish), the command \makeSide

\qOne{What are discrete structures?}
\aOne{A discrete structure refers to a mathematical system that is composed of distinct, separate  elements, as opposed to continuous structures where elements can vary smoothly. Think of a digital clock  vs. an analog clock (where second hand loops around continuously without stopping). Examples of discrete structures include sets with finitely many elements (e.g. the integers 1 to 10), lists, graphs, and logical statements. }


\qTwo{What is discrete mathematics?}
\aTwo{Discrete mathematics is the study of discrete structures and mathematical operations that can be performed upon them.}


\qThree{Why study discrete mathematics?}
\aThree{%It does not directly
%help us write programs. At the same time, 
It is the mathematics underlying almost all of computer science. Here are a few examples: (1)  designing high-speed networks and message routing paths,  (2) finding good algorithms for sorting, (3) performing web searches, (4) analyzing algorithms for correctness and efficiency, (5) formalizing security requirements, and (6) designing cryptographic protocols.
}

\qFour{Where is discrete mathematics used in the MSU computer science curriculum?}
\aFour{Discrete mathematics is used  throughout the curriculum.  Proofs in particular play a critical role in CSCI 338 (Computer Science Theory) and CSCI 432 (Advanced Algorithm Topics).
}



%----------------------------------------------------------------------------------------

\begin{document}


\makeprofile % Print the sidebar

%%%%

\section{Course description}

Mathematical concepts used in computer science with an emphasis on mathematical reasoning and proof techniques. This course covers mathematical thinking (elementary logic, theorems, proof techniques), discrete collections (such as sets and lists),  functions and relations,  counting (including permutations and combinations), and miscellaneous topics (recursions, Big-O notation, graphs).  Writing and understanding proofs will be an important part of this course.

%%%%%
\vspace{0.5cm} 
\section{Textbooks}

{\color{myCOLOR} Primary text}\\
Scheinerman, E. R. (2013). \textit{Mathematics: a discrete introduction.} Third edition.\\
\\
{\color{myCOLOR} Supplemental text}\\
Hamkins, J. D. (2020). \textit{Proof and the Art of Mathematics.}	 MIT Press.

%%%%%
\vspace{0.5cm}
\section{Learning Outcomes}

%use \begin{outline} or \begin{outline}[enumerate] to create a list with subitems. 
\begin{itemize}
\item To understand several central concepts from discrete mathematics that are used throughout computer science;
\item To learn to reason mathematically and communicate ideas in a clear and concise manner;
\item To use induction and other techniques of mathematical proofs.
\end{itemize}


%{\color{myCOLOR} Other}\\
%Any required journal articles and book chapters will be provided on \red{Canvas}. 


%%%%%
\vspace{0.5cm} 
\section{Learning Philosophy}

This course will use an ``inquiry-based learning" approach. This means that classes will not be lecture-oriented. Instead, to quote Dr. Dana Ernst,

\begin{quotation}
\textit{You will be expected to work actively to construct your own understanding of the topics at hand with the readily available help of me and your classmates. Many of the concepts you learn and problems you work on will be new to you and ask you to stretch your thinking. You will experience frustration and failure before you experience understanding. This is part of the normal learning process. If you are doing things well, you should be confused at different points in the semester.}
\end{quotation}

Active learning has been shown to increase student performance in STEM courses (e.g. see \cite{deslauriers2011improved, freeman2014active}). 

%%%%
\vspace{0.2cm}
\section{Class activity}

A typical class meeting will be structured as follows:

\noindent
\begin{minipage}{0.45\textwidth}
Mon/Wed
\begin{itemize}
\item Reading quiz: 5 mins.
\item Mini-lecture: 15 mins
\item Group exercises: 30 mins	
\end{itemize}
\end{minipage}
\hfill
\begin{minipage}{0.45\textwidth}
Fri
\begin{itemize}
\item Reading quiz: 5 mins
\item Problems quiz: 10 mins 
\item Mini-lecture: 5-10 mins
\item Group exercises: 25-30 mins
\end{itemize}
\end{minipage}

%The course structure follows best practices developed by the Carl Wieman Science Education Initiative (CWSEI).

%%%%%
\vspace{0.1cm} 

{\color{myCOLOR} Group exercises}

The group exercises are for collaborative problem-solving.  Students will be split randomly into groups of three to work on problems using the whiteboards (or windows). Mathematics is not a spectator sport!

%%%%%
{\color{myCOLOR} Daily readings}

Before each class meeting, you will be assigned a reading from the textbook. The reading serves preparation for the group exercises. Before some classes, you will be given a short reader response questionnaire (with questions like, "what did you find confusing?") posted as a survey Brightspace.  Each class meeting will begin with a 5-minute reading quiz.


%%%%%
{\color{myCOLOR} Problems quiz}

The problems quiz will be based on the problems given in-class.

% Lecture is to go over the reading quiz, cover things people didn't understand from the current/previous reading. 

%%%%
%%%%% NEW PAGE 

\newpage % Start a new page

\makeSide % Print the FAQ sidebar; To get rid of, simply comment out and uncomment \makeFullPage

% \makeFullPage

\vspace{0.2cm}
\section{Grading}

\begin{itemize}
\item Reading Quizzes (Daily): 25\%  
\item Problems Quizzes (Weekly): 20\% 
\item Project: 15\% 
\item Participation: 15\% 
\item Final: 25\%
\end{itemize}

Grades will be assigned as follows: \\
A: 90-100, B+: 87-90, B: 80-87, C+: 77-80, C: 70-77, D+: 67-70, D: 60-67


%%%%% 
%\vspace{0.5cm}
%\section{Rubric} Quizzes will be graded according to the following rubric.
%\begin{itemize}
%\item (4 points) Correct and well-written.
%\item  (3 points) Good work but some mathematical or writing errors that need addressing.
%\item (2 points) Some good intuition, but there is at least one serious flaw.
%\item (1 point) I don't understand this, but I see that you did work on it.
%\item (0 points) No work is evident.
%\end{itemize}


%%%%
\vspace{0.5cm}
\section{Project} 

The purpose of the project is for students to focus on some aspect of discrete mathematics of particular interest.  Students will work in small groups, choose some aspect of discrete mathematics (that was not covered in class), and develop a 5 minute presentation to give to the class.   For example, a group might give an overview to public key cryptography, or explore how graph theory underlies web search.   More information will be given as the time approaches.


\vspace{0.5cm}
\section{Makeup policy}

Quizzes and participation have no makeups. However, I drop 3 reading quizzes, 3 participation days, and 1 problems quiz with no consequence. Missed material beyond that (with a valid explanation) can be replaced with the final exam grade.

\vspace{0.5cm}
\section{Communication expectations}

For non-personal questions related to course content, please use the Ask Your Instructor form on D2L, so that all students can benefit from the question and answer.   For personal questions, please contact me by e-mail.

%\newpage % Start a new page
%\makeEmptySide

\vspace{0.5cm}
\section{Diversity and Inclusivity Statement}

I consider this classroom to be a place where you will be treated with respect, and I welcome individuals of all ages, backgrounds, beliefs, ethnicities, genders, gender identities, gender expressions, national origins, religious affiliations, sexual orientations, ability - and other visible and non-visible differences. All members of this class are expected to contribute to a respectful, welcoming and inclusive environment for every other member of the class.  Your suggestions about how to improve the value of diversity in this course are encouraged and appreciated. Please let me know ways to improve the effectiveness of the course for you personally or for other students or student groups.
 
\vspace{0.5cm}
\section{Accommodations for Students with Disabilities}

If you are a student with a disability and wish to use your approved accommodations for this course, contact me during my office hours to discuss. Please have your Accommodation Notification available for verification of accommodations. Accommodations are approved through the Office of Disability Services located in 137 Romney Hall.  www.montana.edu/disabilityservices.

\vspace{0.5cm}
\section{Student Conduct}

You are expected to abide by MSU's Code of Student Conduct.
%
%\vspace{0.5cm}
%\section{Land Acknowledgement}
%
%Living in Montana, we are on the ancestral lands of American Indians, including
%the 12 tribal nations that call Montana home today: A’aninin (Gros Ventre),
%Amskapi/Piikani (Blackfeet), Annishinabe (Chippewa/Ojibway), Annishinabe/Métis
%(Little Shell Chippewa), Apsáalooke (Crow), Ktunaxa/Ksanka (Kootenai), Lakota,
%Dakota (Sioux), Nakoda (Assiniboine), Ne-i-yah-wahk (Plains Cree), Qíispé (Pend
%d’Oreille), Seliš (Salish), and Tsétsêhéstâhese/So’taahe (Northern Cheyenne).
%We honor and respect these tribal nations as we live, work, learn, and play in
%this state. To learn more about Montana Indians, consider reading the pamphlet \textit{Essential Understandings Regarding Montana Indians}, available online.



%%%%%%%%%%%%%%%%%%%%%%%%%%%%%%%%%%%%%%%%%%%%%%%%%%%%%%%%%%%%%%%%%%%%%%%%%%%%%
%                COURSE SCHEDULE
%%%%%%%%%%%%%%%%%%%%%%%%%%%%%%%%%%%%%%%%%%%%%%%%%%%%%%%%%%%%%%%%%%%%%%%%%%%%%
\newpage
\makeFullPage
\section{Class Schedule}

%https://github.com/mikewojnowicz/csci246_spring2025/blob/main/LINKS.md

\begin{center}
\begin{tabularx}{\textwidth}{p{2cm}p{2cm}p{8cm}p{9.5cm}} %change the width of the comments by changing these cm measurements. Add/substract columns by adding/deleting p{} sections. 
\arrayrulecolor{myCOLOR}\hline
\multicolumn{4}{l}{\textbf{\textcolor{myCOLOR}{\large MODULE 0: Course Overview}}} \\
\hline
Wed & Jan 15 & Course Overview & \\
%%%%%%%%%%%%%%%%%%%%%%%%%%%%%%%%%%%%%%%%%%% MODULE 1
\arrayrulecolor{myCOLOR}\hline
\multicolumn{4}{l}{\textbf{\textcolor{myCOLOR}{\large MODULE 1: Introduction to mathematical thinking }}} \\
\hline
Fri & Jan 17 &  Theorem & Sec 4 \\
Mon &  Jan 20 & Holiday (Martin L. King Day) & \\
Wed &  Jan 22 & Proof & Sec 5 \\
Fri &  Jan 24 & Counterexample & Sec 6 \\
Mon &  Jan 27 & Boolean Algebra & Sec 7 \\
Wed &  Jan 29 & Multiple Proofs  & JDH CH 2 \\
Fri &  Jan 31 & Induction & JDH CH 4.2, 4.3 (skip Thms. 26-28), 6.6, 6.7 \\
Mon &  Feb 3 & Induction (continued) & JDH CH 4.5-4.6 (skip proof of Thm. 29)  \\
\hline
\multicolumn{4}{l}{\textbf{\textcolor{myCOLOR}{\large MODULE 2: Discrete Collections }}} \\
\hline

Wed &  Feb 5 & Lists & Sec 8 \\
Fri &  Feb 7 & Factorial & Sec 9 \\
Mon &  Feb 10 & Sets & Sec 10 \\
Wed &  Feb 12 & Quantifiers &  Sec 11  \\
Fri &  Feb 14 & Operations on Sets & Sec 12 \\
Mon &  Feb 17 & Holiday (President's Day) & \\
\hline
\multicolumn{4}{l}{\textbf{\textcolor{myCOLOR}{\large MODULE 3: Relations and functions }}} \\
\hline
Wed &  Feb 19 & Intro to Relations and Functions & JDH CH 11.1-11.2\\
Fri &  Feb 21  & Intro to Relations and Functions (continued) & JDH CH 11.5\\
Mon &  Feb 24 & Relations & Sec 14 \\
Wed &  Feb 26 & Equivalence Relations & Sec 15 \\
Fri &  Feb 28 & Partitions & Sec 16 (Optional: JDH Ch. 11.3)\\
Mon &  Mar 3 & Functions & Sec 24 (Skip "Counting functions" \\
 & & & \quad and "Counting functions, again")\\
\hline
\multicolumn{4}{l}{\textbf{\textcolor{myCOLOR}{\large MODULE 4: Counting}}} \\
\hline
Wed &  Mar 5  & Binomial Coefficients (Combinations) & Sec 17  \\
Fri &  Mar 7  & Catch-up day & No reading \\
Mon &  Mar 10  & Inclusion-Exclusion & Sec 19 \\
%Wed &  Mar 12  & Pigeonhole Principle & Sec 25 (Optional: JDH Ch. 5.7) \\
\hline
\multicolumn{4}{l}{\textbf{\textcolor{myCOLOR}{\large MODULE 5: Discrete probability}}} \\
\hline
Wed &  Mar 12 & Intro to Probability (Part 1) & Sec 30 \\
Fri &  Mar 14 & Review Day & No Reading \\
Week &  Mar 17-21 & Spring Break & \\ 
Mon &  Mar 24 &  Intro to Probability  (Part 2) & Sec 31 \\
Wed &  Mar 26 & Conditional Probability /Independence &  Sec 32\\
Fri & Mar 28  &  Random variables & Sec 33 \\
Mon &  Mar 31 & Expectation & Sec 34 \\
\hline
\multicolumn{4}{l}{\textbf{\textcolor{myCOLOR}{\large MODULE 6: Sequences, recurrences, and computational complexity}}} \\
\hline
Wed &  Apr 2 & Recurrence  & Sec 23 \\
Fri &  Apr 4  & Big O Notation & Handout to be posted on Brightspace \\
Mon &  Apr 7  & Master Theorem & Handout to be posted on Brightspace \\
\hline
\multicolumn{4}{l}{\textbf{\textcolor{myCOLOR}{\large MODULE 7: Graph theory}}} \\
\hline
Wed &  Apr 9 & Fundamentals of Graph Theory & Sec 47 \\
Fri &  Apr 11 & Subgraphs & Sec 48 \\ 
Mon &  Apr 14  & Connection & Sec 49 \\ 
Wed &  Apr 16 & Trees & Sec 50 \\ 
%% Mon Mar
%Wed &  Apr 2 & Recurrence  & Sec 23 \\
%Fri &  Apr 4 & Complexity & Handout to be posted on Brightspace \\
%Mon &  Apr 7 & Graph theory & JDH CH 12 \\
%Wed &  Apr 9 & Graph theory (continued) & JDH CH 12 \\
\hline
\multicolumn{4}{l}{\textbf{\textcolor{myCOLOR}{\large MODULE 8: Group Presentations}}} \\
\hline
Fri &  Apr 18 & Holiday (University Day / Mike's Birthday) & \\
Mon &  Apr 21 & Prepare Group Presentations & \\
Wed &  Apr 23 & Prepare Group Presentations & \\
Fri &  Apr 25 & Group Presentations & \\
Mon &  Apr 28 & Group Presentations  & \\
Wed &  Apr 30 & Group Presentations & \\
Fri &  May 2 & Final Preparation &\\
% \hline
% Week & Topic & Readings \\ \hline 
%%%Alternatively, instead of Week #, you can do Class date for meeting
%Week 1 & History of the Earth - Fish Remix & Friedman, M. \& Salland, L.C. (2012). Five hundred million years of extinction and recovery: A Phanerozoic survey of large-scale diversity patterns in fishes. \textit{Palaeontology}, 55(4):707-742 \\
%
%& Stem \& Extant Agnathans \& Gnathostomes & DOF Ch. 11, pp. 169-179; Ch. 13, pp. 231-240  \\
%& & Brazeau, M.D. \& Friedman, M. (2015). The origin and early phylogenetic history of jawed vertebrates. \textit{Nature}, 520(7548): 490-497.\\
%\arrayrulecolor{maingray}\hline
%\arrayrulecolor{myCOLOR}\hline
%\multicolumn{2}{l}{\textbf{\textcolor{myCOLOR}{\large MODULE 3: There Goes the Neighborhood }}} \\
%\hline 
\hline 
\end{tabularx}
Note: The class schedule is tentative; it is subject to change as the course progresses.
\end{center}

\vspace{1cm}
\section{External Resources}

For a list of external resources on discrete mathematics, click  \href{https://github.com/mikewojnowicz/csci246_spring2025/blob/main/LINKS.md}{here}.  This list will be updated throughout the semester.

\vspace{0.5cm}
%\newpage
%\makeFullPage
\bibliography{references}
\bibliographystyle{unsrt}


\end{document} 


