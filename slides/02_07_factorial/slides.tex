\documentclass[10pt]{beamer}

%%%
% PREAMBLE FOR THIS DOC 
%%%
%https://tex.stackexchange.com/questions/68821/is-it-possible-to-create-a-latex-preamble-header
\usepackage{/Users/miw267/Repos/csci246_spring2025/slides/preambles/beamer_preamble_for_CSCI246}



%%% TRY TO RESHOW TOC AT EACH SECTION START (with current section highlighted)
% Reference: https://tex.stackexchange.com/questions/280436/how-to-highlight-a-specific-section-in-beamer-toc
\newcommand\tocforsect[2]{%
  \begingroup
  \edef\safesection{\thesection}
  \setcounter{section}{#1}
  \tableofcontents[#2,currentsection]
  \setcounter{section}{\safesection}
  \endgroup
}


%%%% HERES HOW TO DO IT CORRECTLY
% FIRST IN .STY FILE, DO
%\usetheme[sectionpage=none]{metropolis}
% THEN AT EACH SECTION DO
%\begin{frame}{Outline}
%  \tableofcontents[currentsection]	
%\end{frame}



%\setbeamertemplate{navigation symbols}{}
%\setbeamertemplate{footline}[frame number]{}


%%%
% DOCUMENT
%%%

\begin{document}

%\maketitle

%% Title page frame
%\begin{frame}
%    \titlepage 
%\end{frame}





\title{02/07/2025: Factorial}
\author{CSCI 246: Discrete Structures}
\date{Textbook reference: Sec. 9, Scheinerman}

\begin{frame}
    \titlepage 
\end{frame}


\begin{frame}
\footnotesize 
%\red{TODO}

\begin{mygreenbox}[title=Graded Quiz Pickup]
Quizzes are in the front of the room, grouped into four bins (A-G, H-L, M-R, S-Z) by last name. The quizzes are upside down with your last name on the back. Come find yours before, during, or after class.  Only turn the quiz over if it's yours.
\end{mygreenbox} 
\vfill 

\begin{myredbox}[title=Announcements]

\begin{itemize}
\item Wednesday's reading quiz was graded out of 1 point.  45/60 students who took the quiz scored 100\%.
\item Note: The reading quizzes are equally weighted.  The denominator is typically chosen for convenience.  
\end{itemize}

\end{myredbox}

\vfill 


\begin{myyellowbox}[title=Today's Agenda]
\begin{itemize}
	\item Reading and problems quiz  (25 mins)
	\item Mini-lecture ($\approx$ 10 mins)
%	%
%	\begin{itemize}
%	\footnotesize 
%	\item Review induction 
%	\end{itemize}
%	%
	\item Group exercises ($\approx$ 15 mins)
\end{itemize}

\end{myyellowbox}
\vfill 

\end{frame}




\begin{frame}{Today's Quiz}
\footnotesize 

\alert{Reminder:} Please write your last name on the back of your page.
\vfill 

 \begin{myredbox}[title=Reading Quiz (Factorial)]
Evaluate $0!$. Explain your answer.  (Only one explanation is needed.)
\end{myredbox}

\vfill \vfill 
 \begin{mygreenbox}[title=Problems Quiz (Boolean Algebra \& Induction)]
\begin{enumerate}
	\item The \textbf{Fibonacci sequence} is the sequence given by the following recursive rules:
	\[f_0 =0, \qquad \quad f_1 =1, \qquad \quad f_{n+2} = f_n + f_{n+1} \] Show by induction that $f_0 + \cdots + f_n = f_{n+2} -1$ in the Fibonacci sequence.
	\item Is the following a tautology or a contradiction?
		\[ (X \lor Y) \lor (X \lor \lnot Y) \]
 	    Justify your answer using either a truth table or the properties of Boolean operators. 
 	\item (Extra Credit.) Provide a second justification for \#2.
 	\item (Extra Credit.) When does $(X \implies X) \implies X$? Justify your answer.
\end{enumerate}
\end{mygreenbox}
\end{frame}
% Why no properties? Well, didn't fit on the slide. But you can leave that for extra credit, the properties are simple ones we've used a lot, and you don't have to name them. 


\begin{frame}{Solution to \#4}

\textbf{Problem.} When does $(X \implies X) \implies X$? Justify your answer.

\vfill 
\textbf{Solution.} In a group exercise on Boolean Algebra, we showed that $(X \implies X)$ is a tautology (i.e., it is always \texttt{TRUE}).  Hence, the question reduces to when does $\texttt{TRUE} \implies X$? 

Now recall the truth table for implication. Looking only at the cases where the hypothesis (the proposition in the \enquote{if} statement) is true, we have

\begin{table}
\centering
\begin{tabular}{cc|c}
W  & X & $W \implies X$  \\
\hline 
T & T & T  \\
T & F & F \\
\end{tabular}
\end{table}

Hence, $\texttt{TRUE} \implies X$ only when $X$ is true.

We conclude that $(X \implies X) \implies X$ only when $X$ is true. 
\end{frame}


\begin{frame}[standout]
Review List Group Exercises.	
\end{frame}




\begin{frame}
\footnotesize
Group 1: adam.wyszynski,blake.leone,tristan.nogacki\\
Group 2: john.fotheringham,jacob.ruiz1,samuel.mosier\\
Group 3: connor.graville,nolan.scott1,jack.fry\\
Group 4: jonas.zeiler,peter.buckley1,timothy.true\\
Group 5: reid.pickert,james.brubaker,pendleton.johnston\\
Group 6: connor.yetter,lucas.jones6,cameron.wittrock\\
Group 7: conner.reed1,colter.huber,kaden.price\\
Group 8: anthony.mann,joseph.triem,jacob.shepherd1\\
Group 9: aaron.loomis,carver.wambold,peyton.trigg\\
Group 10: zeke.baumann,jakob.kominsky,emmeri.grooms\\
Group 11: luke.donaldson1,ethan.johnson18,derek.price4\\
Group 12: jacob.ketola,luka.derry,samuel.hemmen\\
Group 13: joseph.mergenthaler,connor.mizner,evan.barth\\
Group 14: matthew.nagel,alexander.knutson,lynsey.read\\
Group 15: michael.oswald,griffin.short,erik.moore3\\
Group 16: justice.mosso,owen.obrien,mason.barnocky\\
Group 17: yebin.wallace,evan.schoening,tyler.broesel\\
Group 18: william.elder1,caitlin.hermanson,jett.girard\\
Group 19: devon.maurer,sarah.periolat,julia.larsen\\
Group 20: samuel.rollins,bridger.voss,jada.zorn\\
Group 21: ryan.barrett2,carsten.brooks,micaylyn.parker\\
Group 22: delaney.rubb,jeremiah.mackey,alexander.goetz\\
\end{frame}


\begin{frame}{Group exercises: Factorial}


\begin{enumerate}
\item Your friend Sandy has six different books of classical literature, eight different \textit{romantasy} (i.e. romance fantasy) books, and five different self-help books.
	\begin{itemize}
	\item[a)] In how many different ways can Sandy's books be arranged on a bookshelf?
	\item[b)] In how many different ways can Sandy's books be arranged on the bookshelf if all books of the same category are grouped together? 
	\end{itemize}
\item Evaluate $\frac{100!}{98!}$ without calculating $100!$ or $98!$.
\item Calculate the following products:
	\begin{itemize}
	\item[a)] $\prod_{k=1}^n \frac{k+1}{k}$.  \qquad (Reduce this product to a simpler expression.)
	\item[b)] $\prod_{k=1}^n (2k-1)$.  \qquad  (This product has a simple English-language description.  What is it?)
	\end{itemize}
\end{enumerate}

\end{frame}


\begin{frame}{Solution to group exercise \#1}

\footnotesize 
\textbf{Problem.} Your friend Sandy has six different books of classical literature, eight different \textit{romantasy} (i.e. romance fantasy) books, and five different self-help books.
	\begin{itemize}
	\item[a)] In how many different ways can Sandy's books be arranged on a bookshelf?
	\item[b)] In how many different ways can Sandy's books be arranged on the bookshelf if all books of the same category are grouped together? 
	\end{itemize}
\vfill 

\textbf{Solution.}

	\begin{itemize}
	\item[a)] By Theorem 8.6b from Scheinerman, there are $(6+8+5)! = 19!$ different ways to arrange the books.
	\item[b)] By Theorem 8.6b from Scheinerman, there are $6!$ ways to arrange the classical literature books, $8!$ ways to arrange the romantasy books, and $5!$ ways to arrange the self-help books.   So by the multiplication principle, there are $6!8!5!$ ways to arrange the books, if we assume a particular ordering of the categories (e.g. classical $\rightarrow$ romantasy $\rightarrow$ self-help).  Now by Theorem 8.6b from Scheinerman, there are $3!$ ways to arrange the categories. Hence, by the multiplication principle again, there are  $3!6!8!5!$ ways to arrange the books overall.
	\end{itemize}
\end{frame}


\begin{frame}{Solution to group exercise \#2}


\textbf{Problem.}   Evaluate $\frac{100!}{98!}$ without calculating $100!$ or $98!$.
\vfill 

\textbf{Solution.}

\[ \df{100!}{98!} = \df{100 \cdot 99 \cdot 98!}{98!} = \df{100 \cdot 99 \cdot \cancel{98!}}{\cancel{98!}} = 100 \cdot 99 = 9900.\]
\end{frame}


\begin{frame}{Solution to group exercise \#3}
\footnotesize 

\textbf{Problem.}  Calculate the following products:
	\begin{itemize}
	\item[a)] $\prod_{k=1}^n \frac{k+1}{k}$.  \qquad (Reduce this product to a simpler expression.)
	\item[b)] $\prod_{k=1}^n (2k-1)$.  \qquad  (This product has a simple English-language description.  What is it?)
	\end{itemize}
\vfill 

\textbf{Solution.}
	\begin{itemize}
	\item[a)] 
	\begin{align*}
	\prod_{k=1}^n \frac{k+1}{k} &= \frac{2}{1} \cdot \frac{3}{2} \cdots \frac{n}{n-1} \cdot \frac{n+1}{n} \\
	&= \frac{\cancel{2}}{1} \cdot \frac{\cancel{3}}{\cancel{2}} \cdots \frac{\cancel{n}}{\cancel{n-1}} \cdot \frac{n+1}{\cancel{n}} \\
	&= n+1
	\end{align*}
	\item[b)] 
	\begin{align*}
	\prod_{k=1}^n (2k-1) &= (2 \cdot 1 -1) \cdot (2 \cdot 2 -1) \cdot (2 \cdot 3 -1) \cdots  (2 \cdot n -1) \\
	&= 1 \cdot 3 \cdot 5 \cdots 2n-1
	\end{align*}
	%
	This is the product the first $n$ odd natural numbers. (The $k$-th factor is the $k$-th odd natural number.)
	\end{itemize}

\end{frame}

\end{document}
