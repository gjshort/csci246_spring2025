\documentclass[10pt]{beamer}

%%%
% PREAMBLE FOR THIS DOC 
%%%
%https://tex.stackexchange.com/questions/68821/is-it-possible-to-create-a-latex-preamble-header
\usepackage{/Users/miw267/Repos/csci246_spring2025/slides/preambles/beamer_preamble_for_CSCI246}

\usetikzlibrary{matrix}


%%% TRY TO RESHOW TOC AT EACH SECTION START (with current section highlighted)
% Reference: https://tex.stackexchange.com/questions/280436/how-to-highlight-a-specific-section-in-beamer-toc
\newcommand\tocforsect[2]{%
  \begingroup
  \edef\safesection{\thesection}
  \setcounter{section}{#1}
  \tableofcontents[#2,currentsection]
  \setcounter{section}{\safesection}
  \endgroup
}


%%%% HERES HOW TO DO IT CORRECTLY
% FIRST IN .STY FILE, DO
%\usetheme[sectionpage=none]{metropolis}
% THEN AT EACH SECTION DO
%\begin{frame}{Outline}
%  \tableofcontents[currentsection]	
%\end{frame}



%\setbeamertemplate{navigation symbols}{}
%\setbeamertemplate{footline}[frame number]{}


%%%
% DOCUMENT
%%%

\begin{document}

%\maketitle

%% Title page frame
%\begin{frame}
%    \titlepage 
%\end{frame}




%
%\title{03/12/2025: Intro to Probability (Part 1): \\ \qquad \qquad \qquad Sample Space \& Events}
\title{03/24/2025: Intro to Probability (Part 2)}
\author{CSCI 246: Discrete Structures}
\date{Textbook reference: Sec 31, Scheinerman}

\begin{frame}
    \titlepage 
\end{frame}


\begin{frame}
\footnotesize 
%\begin{mygreenbox}[title=Graded Quiz Pickup]
%Quizzes are in the front of the room, grouped into four bins (A-G, H-L, M-R, S-Z) by last name. The quizzes are upside down with your last name on the back. Come find yours before, during, or after class.  Only turn the quiz over if it's yours.
%\end{mygreenbox} 
%\vfill 

\begin{myredbox}[title=\text{Announcement: This Friday's problems quiz}]

The Problems Quiz this Friday 03/28 will cover the following topics:
\begin{itemize}
\item Binomial Coefficients (see	 slide deck for 03-05)
\item Inclusion-Exclusion (see slide deck for 03-10)
\item Intro to Probability (see slide decks for 03-12 and 03-24)
\end{itemize}
\end{myredbox}

\vfill 


\begin{myyellowbox}[title=Today's Agenda]
\begin{itemize}
	\item Reading quiz (5 mins)
	\item Mini-lecture ($\approx$ 20 mins)
	\item Group exercises ($\approx$ 20 mins)
\end{itemize}


\end{myyellowbox}
\vfill 

\end{frame}





\begin{frame}[standout]
Reading quiz
\end{frame}

\begin{frame}

\begin{mygreenbox}[title=\text{Reading Quiz (Events)}]
Let $A$ and $B$ be events in a probability space $(S,P)$.  

\begin{enumerate}
\item State whether the equation below is true or false:
\[ P(A \cup B) = P(A) + P(B) \]	
\item If the equation is true, provide a proof.  If the equation is false, provide the correct equation.
\end{enumerate}

\end{mygreenbox}
	
\end{frame}


\begin{frame}[standout]
Introduction to Probability: \\
Review Of Samples and Events
\end{frame}

\begin{frame}[standout]
Applications \\
Of The Equally Likely Probability Formula
\end{frame}


\begin{frame}
\begin{mygreenbox}[title=\text{Example: Ten Dice}]
Ten dice are tossed.  What is the probability that none of the dice shows the number one? \\

\pause 
Let $S$ be the sample space.  That is,
\begin{align*}
 S&=\bigg\{(\texttt{1,1,1,1,1,1,1,1,1,1,1}), (\texttt{1,1,1,1,1,1,1,1,1,2}), \hdots,  \\
 & \qquad \qquad (\texttt{6,6,6,6,6,6,6,6,6,6,6,6})\bigg\}	
\end{align*}
%
Let $E \subset S$ be the event that none of the dice shows the number one. Then by the equally likely probability formula, 
\begin{align*}
P(E) = 	\frac{|E|}{|S|} 
\end{align*}
\pause 
Now $|S|=6^{10}$ and $|E|=5^{10}$.  
\pause 
Hence,
%
\begin{align*}
P(E) = 	\frac{|E|}{|S|} = \bigg(\frac{5}{6}\bigg)^{10} \approx 0.1615.
\end{align*}
\end{mygreenbox}

\end{frame}


\begin{frame}
\begin{mygreenbox}[title=\text{Example: Coin tossing}]
A coin is tossed five times.  What is the probability that we see exactly two heads?  \\

\pause 
Let $S$ be the sample space.  That is,
 \[ S=\bigg\{\texttt{(H,H,H,H,H)}, \texttt{(H,H,H,H,T)}, \hdots, \texttt{(T,T,T,T,T)} \bigg\}. \]
Let $E \subset S$ be the event that exactly two heads are observed. Then by the equally likely probability formula, 
\begin{align*}
P(E) = 	\frac{|E|}{|S|} 
\end{align*}
\pause 
Now $|S|=2^{5}$ and $|E|=\binom{5}{2}$. 
\pause
Hence,
%
\begin{align*}
P(E) = 	\frac{|E|}{|S|} = \frac{\binom{5}{2}}{2^{5}} = \frac{10}{32} \approx 0.3125.
\end{align*}
\end{mygreenbox}

\end{frame}

\begin{frame}
\small 
\begin{mygreenbox}[title=\text{Example: Four of a kind}]
A poker hand is called a \textit{four of a kind} if four of the five cards show the same value (e.g. all 7s or all kings).  What is the probability that a poker hand is a four of a kind?  \\

\pause 
Let $S$ be the sample space.  That is, 
\[S=\bigg\{ \set{A\spadesuit 2 \spadesuit 3 \spadesuit 4 \spadesuit 5 \spadesuit}, \;   \hdots, \; \set{9 \heartsuit 10 \heartsuit J \heartsuit Q \heartsuit K \heartsuit} \bigg\}. \]
Let $E \subset S$ be the event that the poker hand is a four of a kind. Then by the equally likely probability formula, 
\begin{align*}
P(E) = 	\frac{|E|}{|S|} 
\end{align*}
\pause 
Now $|S|=\binom{52}{5}$ and $|E|=13 \times 48$. 
\pause
Hence,
%
\begin{align*}
P(E) = 	\frac{|E|}{|S|} = \frac{13 \times 48}{\binom{52}{5}} \approx 0.00024.
\end{align*}
\vfill 
\textbf{Final Answer}: The probability of getting a four of a kind in a 5-card poker hand is approximately 0.00024, or about 1 in 4,167 hands.
\end{mygreenbox}

\end{frame}

\begin{frame}[standout]
Warning: \\
Outcomes Aren't Always Equally Likely
\end{frame}

\begin{frame}

\begin{myredbox}[title=\text{Warning: Outcomes Aren't Always Equally Likely}]
Consider the probability space $(S,P)$ where the sample space
\[ S=\set{A,B,C,D,F}\]
is your possible grades in CSCI 246. Suppose that you have been performing well in the class so far, and the probability of you obtaining each grade is given by
\[ P(A)= .50, \; P(B)= .30, \;  P(C)= .10, \;  P(D)= .07, \;  P(F)= .03 \]
Suppose $E=\set{D,F}$ is the event that you get a grade that you or your parents might consider unsatisfactory .
Then
\[ P(E) = \sum_{s \in E} P(s) = P(D) + P(F) = .07 + .03 = .10\]

Note that 
\[ P(E) \neq \frac{|E|}{|S|} = \frac{2}{5} = .40\]
\end{myredbox}

\end{frame}




\begin{frame}[standout]
Some Basic Properties of Probability	
\end{frame}


\begin{frame}
\footnotesize 
\begin{mygreenbox}[title=\text{Proposition: Basic Properties of Probability}]
Let $(S,P)$ be a probability space. Then
\begin{itemize}
	\item[(a)] (Probability of entire space.) $P(S)=1$. 
	\item[(b)] (Probability of empty set.)  $P(\emptyset)=0$. 
	\item[(c)] (Probability of complement.) $P(\overline{A}) = 1-P(A)$, \\
		where $\overline{A} \defeq S-A$ is called the \textit{complement} of $A$ (i.e. A \underline{didn't} happen).
	\item[(d)]  (General addition rule.)  $P(A \cup B) = P(A) + P(B) - P(A \cap B)$, \\
	where $A$ and $B$ be events.
\end{itemize}
\end{mygreenbox}

\pause 
\vfill 
\begin{myredbox}[title=\text{Remark: General Addition Rule As Inclusion-Exclusion}]
Recall the inclusion-exclusion formula (with two sets):
\[ |A \cup B| = |A| + |B| - |A \cap B|\]
The general addition rule is this same formula, but where we measure size using probabilities $P(\cdot)$ instead of cardinality $|\cdot|$.
\end{myredbox}

\pause 
\vfill 
\begin{myyellowbox}[title=\text{Additional properties}]
As a consequence of the  general addition rule, we also have
\begin{itemize}
	\item[(e)] (Subadditivity.) $P(A \cup B) \leq P(A) + P(B)$. 
	\item[(f)] (Additivity.) If $A \cap B = \emptyset$, then  $P(A \cup B) = P(A) + P(B)$.
\end{itemize}
\end{myyellowbox}

\end{frame}


\begin{frame}
\footnotesize 
\begin{mygreenbox}[title=\text{Example: Drawing a Card from a Deck}]
You draw a single card from a standard 52-card deck. What is the probability that the card is either a heart or a face card? \\
\\
\pause 
Let
\begin{itemize}
\item A = the event that the card is a \textbf{heart} $\rightarrow P(A) = 1/4$.\\
\item B = the event that the card is a \textbf{face card} (Jack, Queen, or King of any suit) $\rightarrow P(B) = 3/13$. 
\end{itemize}

\pause 
Note that some cards belong to \textbf{both} events: the \textbf{face cards that are also hearts} (Jack, Queen, and King of hearts). \\


The overlap $A \cap B$ satisfies
\[P(A \cap B)  = \frac{3}{52}.\]

\pause 
To find $P(A \cup B)$, we use the \textbf{general addition rule}.
%
\begin{align*}
P(A \cup B) &= P(A) + P(B) - P(A \cap B) \\
&= \frac{1}{4} + \frac{3}{13} - \frac{3}{52} \\
&= \frac{11}{26} \approx 0.42
\end{align*}
 	
\pause 
\textbf{Final answer.} The probability of drawing either a heart or a face card is $\frac{11}{26}$, which is \textbf{less than} the sum of the individual probabilities because we subtract the overlap.
\end{mygreenbox}

	
\end{frame}




\begin{frame}[standout]
Group exercises
\end{frame}

\begin{frame}
\footnotesize 
\vfill 
\begin{columns}
\begin{column}{0.33\textwidth}
aaron.loomis: 1 \\ 
adam.wyszynski: 13 \\ 
alexander.goetz: 5 \\ 
alexander.knutson: 13 \\ 
anthony.mann: 15 \\ 
blake.leone: 16 \\ 
bridger.voss: 6 \\ 
caitlin.hermanson: 11 \\ 
cameron.wittrock: 8 \\ 
carsten.brooks: 11 \\ 
carver.wambold: 18 \\ 
colter.huber: 11 \\ 
conner.reed1: 20 \\ 
connor.mizner: 19 \\ 
connor.yetter: 15 \\ 
derek.price4: 2 \\ 
devon.maurer: 4 \\ 
emmeri.grooms: 1 \\ 
erik.moore3: 17 \\ 
ethan.johnson18: 8 \\ 
evan.barth: 5 \\\end{column}
\begin{column}{0.33\textwidth}
evan.schoening: 21 \\ 
griffin.short: 4 \\ 
jack.fry: 19 \\ 
jacob.ketola: 10 \\ 
jacob.ruiz1: 3 \\ 
jacob.shepherd1: 21 \\ 
jada.zorn: 9 \\ 
jakob.kominsky: 9 \\ 
james.brubaker: 3 \\ 
jeremiah.mackey: 15 \\ 
jett.girard: 12 \\ 
john.fotheringham: 1 \\ 
jonas.zeiler: 17 \\ 
joseph.mergenthaler: 8 \\ 
joseph.triem: 12 \\ 
julia.larsen: 2 \\ 
justice.mosso: 20 \\ 
kaden.price: 18 \\ 
lucas.jones6: 16 \\ 
luka.derry: 14 \\ 
luke.donaldson1: 17 \\\end{column}
\begin{column}{0.33\textwidth}
lynsey.read: 7 \\ 
mason.barnocky: 13 \\ 
matthew.nagel: 12 \\ 
micaylyn.parker: 4 \\ 
michael.oswald: 2 \\ 
nolan.scott1: 14 \\ 
owen.obrien: 5 \\ 
pendleton.johnston: 7 \\ 
peter.buckley1: 6 \\ 
reid.pickert: 19 \\ 
ryan.barrett2: 9 \\ 
samuel.hemmen: 14 \\ 
samuel.mosier: 10 \\ 
samuel.rollins: 7 \\ 
sarah.periolat: 18 \\ 
timothy.true: 3 \\ 
tristan.nogacki: 16 \\ 
tyler.broesel: 10 \\ 
william.elder1: 6 \\ 
yebin.wallace: 20 \\ 
zeke.baumann: 21 \\\end{column}
\end{columns}
\vfill
\end{frame}


\begin{frame}{Group exercises}
\small
\begin{enumerate} \small
	\item (More poker hands.) Give the probability of obtaining the various hands below in a game of 5-card poker.
	\begin{itemize}	 \footnotesize 
	\item[a.] A \textbf{three of a kind}. This means three cards of the same rank and two other cards of different ranks, such as three 10s, a 7, and a jack.
	\item[b.] A \textbf{full house.} This is three cards with one common rank and two other cards of another common rank, such as three queens and two 4s.
	\item[c.] A \textbf{single pair.} This is two cards with the same rank and three other cards with three other ranks, such as two 9s, a king, an 8, and a 5.
	\item[d.] \textbf{Two pairs.} This is two cards with one common rank, two more cards with another common rank, and a third card with yet another rank, such as two jacks, two 8s, and a 3.
	\item[e.] A \textbf{flush.}  This means all five cards have the same suit.
	\end{itemize}
	\item Prove Scheinerman Proposition 31.8.   You may assume the the general addition rule (Scheinerman Proposition 31.7) holds.
	\item Three dice are rolled.  What is the probability that... (a) ... none of the dice shows 1? (b)... at least one die shows 1? (c) ... at least one die shows 2? (d) ... none of the dice shows 1 or 2? (e) ... at least one die shows 1 or at least one die shows 2 (or both)? (f) .... at least one die shows 1 and at least one die shows 2?
\end{enumerate}

\end{frame}


\begin{frame}{Solution to Group Exercise \#1a}
\footnotesize 
 \colorbox{blue!30}{\textbf{Problem.}} What is the probability of getting a \textbf{three of a kind} in a 5-card poker hand? 
 \vfill 

 \colorbox{green!30}{\textbf{Solution.}} Let $S$ be the sample space.  That is, 
\[S=\bigg\{ \set{A\spadesuit 2 \spadesuit 3 \spadesuit 4 \spadesuit 5 \spadesuit}, \;   \hdots, \; \set{9 \heartsuit 10 \heartsuit J \heartsuit Q \heartsuit K \heartsuit} \bigg\}. \]
Let $E \subset S$ be the event that the poker hand is a three of a kind. Then by the equally likely probability formula, 
\begin{align*}
P(E) = 	\frac{|E|}{|S|} = \frac{\text{\# of three of a kind poker hands}}{\text{\# of poker hands}}
\end{align*}
Now $|S|=\binom{52}{5} =  2,598,960$, so we just need to find $|E|$.  We proceed in steps:

\begin{enumerate} \footnotesize 
\item \textbf{Choose the rank} for the three matching cards 	(e.g., three Kings or three 7s).  There are 13 possible ranks.
\item \textbf{Select 3 of the 4 suits} for that rank:  $\binom{4}{3}=4$.
\item  \textbf{Choose two kicker cards} (the remaining two cards that don't match the three of a kind rank). 
	\begin{itemize} \footnotesize 
	\item[a.] 	\textbf{Choose 2 different ranks} from the remaining 12 ranks: $\binom{12}{2} = 66$ 
	\item[b.] \textbf{Select 1 suit} for each chosen rank: $\binom{4}{1} \times \binom{4}{1} = 16$
	\end{itemize}
\item \textbf{Combine the choices}:  $|E|=13 \times 4 \times 66 \times 16 = 54,912$.
\end{enumerate}
\vspace{-.1cm}
\alert{\textbf{Final Answer.}} The probability of getting a three of a kind in a 5-card poker hand is $\frac{54,912}{2,598,960} \approx 0.021128$, or about 1 in 47 hands.
\end{frame}

\begin{frame}{Solution to Group Exercise \#1b}
\footnotesize 
 \colorbox{blue!30}{\textbf{Problem.}} What is the probability of getting a \textbf{full house} in a 5-card poker hand? 
\vfill 

 \colorbox{green!30}{\textbf{Solution.}} Let $S$ be the sample space.  That is, 
\[S=\bigg\{ \set{A\spadesuit 2 \spadesuit 3 \spadesuit 4 \spadesuit 5 \spadesuit}, \;   \hdots, \; \set{9 \heartsuit 10 \heartsuit J \heartsuit Q \heartsuit K \heartsuit} \bigg\}. \]
Let $E \subset S$ be the event that the poker hand is a full house. Then by the equally likely probability formula, 
\begin{align*}
P(E) = 	\frac{|E|}{|S|} = \frac{\text{\# of full house poker hands}}{\text{\# of poker hands}}
\end{align*}
Now $|S|=\binom{52}{5} =  2,598,960$, so we just need to find $|E|$.   We proceed in steps:

\begin{enumerate} \footnotesize 
\item \textbf{Choose the rank} for the three matching cards 	(e.g., three Kings or three 7s).  There are 13 possible ranks.
\item \textbf{Select 3 of the 4 suits} for that rank:  $\binom{4}{3}=4$.
\item \textbf{Choose the rank} for the two matching cards 	(e.g., two Queens or two 6s).  There are 12 possible ranks which differ from the three-of-a-kind rank.
\item \textbf{Select 2 of the 4 suits} for that rank:  $\binom{4}{2}=6$.
\item \textbf{Combine the choices}:  $|E|=13 \times 4 \times 12 \times 6 = 3,744$.
\end{enumerate}

\alert{\textbf{Final Answer.}} The probability of getting a three of a kind in a 5-card poker hand is $\frac{3,744}{2,598,960} \approx 0.00144$, or about 1 in 694 hands.
\end{frame}

\begin{frame}{Solution to Group Exercise \#1c}
\footnotesize 
 \colorbox{blue!30}{\textbf{Problem.}} What is the probability of getting a \textbf{single pair} in a 5-card poker hand?
\vfill 

 \colorbox{green!30}{\textbf{Solution.}} Let $S$ be the sample space.  That is, 
\[S=\bigg\{ \set{A\spadesuit 2 \spadesuit 3 \spadesuit 4 \spadesuit 5 \spadesuit}, \;   \hdots, \; \set{9 \heartsuit 10 \heartsuit J \heartsuit Q \heartsuit K \heartsuit} \bigg\}. \]
Let $E \subset S$ be the event that the poker hand is a single pair. Then by the equally likely probability formula, 
\begin{align*}
P(E) = 	\frac{|E|}{|S|} = \frac{\text{\# of single pair poker hands}}{\text{\# of poker hands}}
\end{align*}
Now $|S|=\binom{52}{5} =  2,598,960$, so we just need to find $|E|$.  We proceed in steps:

\begin{enumerate} \footnotesize 
\item \textbf{Choose the rank} for the two matching cards 	(e.g., two Kings or two 7s).  There are 13 possible ranks.
\item \textbf{Select 2 of the 4 suits} for that rank:  $\binom{4}{2}=6$.
\item  \textbf{Select 3 other ranks} for the remaining 3 cards (out of the 12 ranks which differ from the single pair rank): $\binom{12}{3}=220$. 
\item \textbf{Select 1 suit for each of the 3 different ranks}: $\binom{4}{1} \times \binom{4}{1} \times \binom{4}{1} = 64.$
\item \textbf{Combine the choices}:  $|E|=13 \times 6 \times 220 \times 64 = 1,098,240$.
\end{enumerate}
\vspace{-.1cm}
\alert{\textbf{Final Answer.}} The probability of getting a single pair in a 5-card poker hand is $\frac{1,098,240}{2,598,960} \approx 0.42257$, or about 1 in 2.37 hands.
\end{frame}



\begin{frame}{Solution to Group Exercise \#1d}
\footnotesize 
 \colorbox{blue!30}{\textbf{Problem.}} What is the probability of getting \textbf{two pairs} in a 5-card poker hand?
\vfill 

 \colorbox{green!30}{\textbf{Solution.}} Let $S$ be the sample space of poker hands. 
 % That is, 
%\[S=\bigg\{ \set{A\spadesuit 2 \spadesuit 3 \spadesuit 4 \spadesuit 5 \spadesuit}, \;   \hdots, \; \set{9 \heartsuit 10 \heartsuit J \heartsuit Q \heartsuit K \heartsuit} \bigg\}. \]
Let $E \subset S$ be the event that the poker hand is two pairs. Then by the equally likely probability formula, 
\begin{align*}
P(E) = 	\frac{|E|}{|S|} = \frac{\text{\# of two pair poker hands}}{\text{\# of poker hands}}
\end{align*}
Now $|S|=\binom{52}{5} =  2,598,960$, so we just need to find $|E|$.  We proceed in steps:

\begin{enumerate} \footnotesize 
\item \textbf{Choose the rank} for the first pair 	(e.g., two Kings or two 7s).  There are 13 possible ranks.
\item \textbf{Select 2 of the 4 suits} for that rank:  $\binom{4}{2}=6$.
\item \textbf{Choose the rank} for the second pair.  There are 12 ranks which differ from the first pair rank.
\item \textbf{Select 2 of the 4 suits} for that rank:  $\binom{4}{2}=6$.
\item \textbf{Choose the rank} for the fifth card.  There are 11 ranks which differ from those of the two pairs.
\item \textbf{Select 1 of the 4 suits} for that rank:  $\binom{4}{1}=4$.
\item \textbf{Combine the choices}:  $|E|=13 \times 6 \times 12 \times 6 \times 11 \times 4 = 123,552$.
\end{enumerate}
\vspace{-.1cm}
\alert{\textbf{Final Answer.}} The probability of getting a single pair in a 5-card poker hand is $\frac{123,552}{2,598,960} \approx 0.0475$, or about 1 in 21 hands.
\end{frame}

\begin{frame}{Solution to Group Exercise \#1e}
\footnotesize 
 \colorbox{blue!30}{\textbf{Problem.}} What is the probability of getting a \textbf{flush} in a 5-card poker hand?
\vfill 
 \colorbox{red!30}{\textbf{Remark.}} A \textbf{flush} consists of five cards of the \textit{same suit}, but \textit{not in sequential order}. For example: $2\spadesuit 5\spadesuit 8\spadesuit J\spadesuit Q\spadesuit$.  If the cards are in sequence, it would be a \textbf{straight flush} instead.  
 
\vfill 

 \colorbox{green!30}{\textbf{Solution.}} Let $S$ be the sample space of poker hands.  
 %That is, 
%\[S=\bigg\{ \set{A\spadesuit 2 \spadesuit 3 \spadesuit 4 \spadesuit 5 \spadesuit}, \;   \hdots, \; \set{9 \heartsuit 10 \heartsuit J \heartsuit Q \heartsuit K \heartsuit} \bigg\}. \]
Let $E \subset S$ be the event that the poker hand is a flush.  We write $E=A-B$, where $A$ is the number of flushes (including straight flushes), and $B$ is the number of straight flushes.  Hence, $|E| = |A|-|B|$. 
%Then by the equally likely probability formula, $P(E) = 	\frac{|E|}{|S|} $. 



\begin{minipage}[t]{0.45\textwidth}
\scriptsize 
To find $|A|$:
\begin{enumerate}
\item \textbf{Choose the suit}: There are 4 suits.  
\item \textbf{Select 5 cards from the suit}: $ \binom{13}{5} = 1,287$.
\item \textbf{Combine the choices}: $|A| = 4 \times 1287 = 5,148$.
\end{enumerate}
\end{minipage} %
\hfill
\begin{minipage}[t]{0.45\textwidth}
\scriptsize 
To find $|B|$:
\begin{enumerate} 
\item \textbf{Choose the suit}: There are 4  suits.  
\item \textbf{Select the straight flush in the suit}: For each suit, there are 10 straight flushes, (from $A-2-3-4-5$ to $10-J-Q-K-A$).
\item \textbf{Combine the choices}: $|B| = 4 \times 10 = 40$.
\end{enumerate}
\end{minipage}

\footnotesize
Hence, the number of valid flush hands is  $ |E| = |A|-|B| = 5,148 - 40 = 5,108$.

\vspace{-.1cm}
\alert{\textbf{Final Answer.}} The probability of getting a flush is $\frac{5,108}{2,598,960} \approx 0.001965$, or about 1 in 509 hands.
\end{frame}



\begin{frame}{Solution to Group Exercise \#2}
\scriptsize  
 \colorbox{blue!30}{\textbf{Problem.}} A Let $(S,P)$ be a probability space. Then
\begin{itemize}
	\item[(a)] (Probability of entire space.) $P(S)=1$. 
	\item[(b)] (Probability of empty set.)  $P(\emptyset)=0$. 
	\item[(c)] (Probability of complement.) $P(\overline{A}) = 1-P(A)$, \\
		where $\overline{A} \defeq S-A$ is called the \textit{complement} of $A$ (i.e. A \underline{didn't} happen).
	\item[(d)]  (General addition rule.)  $P(A \cup B) = P(A) + P(B) - P(A \cap B)$, \\
	where $A$ and $B$ be events.
	\item[(e)] (Subadditivity.) $P(A \cup B) \leq P(A) + P(B)$. 
	\item[(f)] (Additivity.) If $A \cap B = \emptyset$, then  $P(A \cup B) = P(A) + P(B)$.
\end{itemize}
Assume (d) above and prove the rest.

 \vfill
  \colorbox{green!30}{\textbf{Solution.}}
 \begin{itemize}
	\item[(a)]  By the defs. of probability space and event, $P(S)= \sum_{s \in S} P(s) = 1.$ 
	\item[(b)]  By the defs. of probability space and event,  $P(\emptyset)=0 = \sum_{s \in \emptyset} P(s) = 0$. 
	\item[(c)] Using the general addition rule with $B=\overline{A}$:
	%
	\begin{align*}
	P(A \cup \overline{A}) &= P(A) + P(\overline{A}) - P(A \cap \overline{A}) \\
	\implies P(S) &= P(A) + P(\overline{A}) - P(\emptyset) \\
	\implies 1 &= P(A) + P(\overline{A}) - 0 && \scripttext{by (a), (b)}
	\end{align*}
	%
	Algebra gives the result.
	\item[(e)] This holds by the general addition rule, noting that $P(A \cap B) \geq 0$.
	\item[(f)] If $A \cap B = \emptyset$, then $P(A \cap B) =0$ by part (b).  Hence, the result follows by the general addition rule.
\end{itemize}

\end{frame}

\begin{frame}{Solution to Group Exercise \#3}
\scriptsize 
 \colorbox{blue!30}{\textbf{Problem.}} Three dice are rolled.  What is the probability that...
    \begin{columnsonlytextwidth}
    \begin{column}{0.35\textwidth}
        \begin{itemize} \tiny 
		\item[a.)] ... none of the dice shows 1? 
		\item[b.)] ... at least one die shows 1? 
		\item[c.)] ... at least one die shows 2?
        \end{itemize}
    \end{column}
    \begin{column}{0.55\textwidth}
        \begin{itemize}  \tiny 
		\item[d.)] ... none of the dice shows 1 or 2?
		\item[e.)]  ... at least one die shows 1 or at least one die shows 2 (or both)? 
		\item[f.)] .... at least one die shows 1 and at least one die shows 2?
        \end{itemize}
    \end{column}
    \end{columnsonlytextwidth}
 \vfill 
  \colorbox{green!30}{\textbf{Solution.}}  
\begin{align*}
\text{Define the following events} &&  A &= \text{at least one die shows 1} &\quad  B &= \text{at least one die shows 2} \\
 \text{So the complementary events are}&&  \overline{A} &= \text{none of the dice show 1} &\quad   \overline{B} &= \text{none of the dice show 2}
\end{align*}
%
\vspace{-.5cm}
 \begin{itemize}
\item[a.)]  The outcomes in event $\overline{A}$ are the lists of length 3 with outcomes chosen from the set $\set{2,3,4,5,6}$.  Hence $|A| = 5^3$. By the equally likely probability formula,  $P(\overline{A}) = \frac{|A|}{|S|} = \frac{5^3}{6^3} \approx  0.5787$.   
\item[b.)] By the probability of a complementary event, $P(A) =  1- P(\overline{A}) \approx 1-0.5787 = 0.4213$. 
\item[c.)] $P(B) = P(A) \approx 0.4213$.
\item[d.)] The outcomes in event $\overline{A} \cap \overline{B}$ are the lists of length 3 with outcomes chosen from the set $\set{3,4,5,6}$.   Hence $|\overline{A} \cap \overline{B}| = 4^3$. By the equally likely probability formula,  $P(\overline{A} \cap \overline{B}) = \frac{|\overline{A} \cap \overline{B}|}{|S|} = \frac{4^3}{6^3} \approx  0.2962$. 
\item[e.)] By the probability of a complementary event, $P(A \cup B) =  1- P(\overline{A \cup B})$.   By DeMorgan's laws for sets (or intuition), $\overline{A \cup B}  = \overline{A} \cap \overline{B}$. Hence, using part (d), $P(\overline{A \cup B}) = 0.2962$, and so $P(A \cup B) =  1- P(\overline{A \cup B}) \approx 0.7038$.
\item[f.)] By the general addition law, $P(A \cap B) = P(A) + P(B) - P(A \cup B)$. Filling in results from (b), (c), and (e), we find $P(A \cap B) \approx 0.4213 + 0.4213 - 0.7038 = 0.1388$. 
 \end{itemize} 

\end{frame}



\end{document}
